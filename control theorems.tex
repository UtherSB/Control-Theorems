\documentclass[12pt]{article}
\usepackage{amsmath}
\usepackage{amsthm}
\usepackage{amssymb}
\usepackage{graphicx}
\usepackage{multicol}
\usepackage[dvips,letterpaper]{geometry} % Margins.
\usepackage{hyperref}
\usepackage{comment}

%\makeatletter % Need for anything that contains an @ command 
%\renewcommand{\maketitle} % Redefine maketitle to conserve space
%{ \begingroup \vskip 10pt \begin{center} \LARGE {\emph \@title}
%	\vskip 40pt \large \@author \vskip 10pt \@date \end{center}
% \vskip 10pt \endgroup \setcounter{footnote}{0} }
%\makeatother % End of region containing @ commands

\let\underdot=\d % rename builtin command \d{} to \underdot{}
\renewcommand{\d}[2]{\frac{d #1}{d #2}} % for derivatives
\newcommand{\dd}[2]{\frac{d^2 #1}{d #2^2}} % for double derivatives
\newcommand{\pd}[2]{\frac{\partial #1}{\partial #2}} % for partial derivatives
\newcommand{\pdd}[2]{\frac{\partial^2 #1}{\partial #2^2}} % for double partial derivatives
\newcommand{\pdc}[3]{\left( \frac{\partial #1}{\partial #2}\right)_{#3}} % for thermodynamic partial derivatives
\newcommand{\avg}[1]{\langle #1 \rangle}
\newcommand{\ket}[1]{| #1 \rangle}
\newcommand{\bra}[1]{\langle #1 |}
\newcommand{\braket}[2]{\langle #1 | #2 \rangle} % for Dirac brackets
\newcommand{\ketbra}[2]{| #1 \rangle\langle #2 |}
\newcommand{\matrixel}[3]{\langle #1 | #2 | #3 \rangle} % for Dirac matrix elements

\def\dbar{{\mathchar'26\mkern-12mu d}}

\newtheorem{theorem}{Theorem}
\newtheorem{proposition}[theorem]{Proposition}
\newtheorem{lemma}[theorem]{Lemma}
\theoremstyle{plain}
\newtheorem{definition}[theorem]{Definition}
\theoremstyle{remark}
\newtheorem*{remark}{Remark}
\theoremstyle{plain}
\newtheorem{corollary}[theorem]{Corollary}
\newtheorem*{example}{Example}

\DeclareMathOperator{\SL}{SL}
\DeclareMathOperator{\SO}{SO}
\DeclareMathOperator{\SU}{SU}
\DeclareMathOperator{\Spin}{Spin}
\DeclareMathOperator{\Oh}{O}
\DeclareMathOperator{\eS}{S}

\usepackage{bbm}
\usepackage{enumerate}
\newcommand{\behaviour}{(\Xi,\widetilde{\mathcal{M}}, P)}
\DeclareMathOperator{\proj}{proj}
\DeclareMathOperator{\st}{s.t.}
\DeclareMathOperator{\Span}{span}
\DeclareMathOperator{\Tr}{Tr}

\begin{document}
\title{Bilinear control theory}
\author{Uther Shackerley-Bennett}
\maketitle

\tableofcontents

\section{The Definitions}


\section{The Theorems}
$m$ denotes the number of control fields, $n$ denotes the number of modes of the system. The control functions are usually either unspecified or constrained to $u \in \mathbb{R}$. By symmetric controls I mean that the set of controls parameters is equal to its negative.

\subsection{Normal Accessibility}
$m=m$, $n=n$, any controls. Necessary and sufficient condition for control on a Lie group. p154 JurdjevicGeo or my pdf.

\subsection{Recursivity theorems}
\subsubsection{Neutrality}
$m=m$, $n=n$, any controls. Sufficient condition for control on a Lie group. Elliott p97.
\begin{itemize}
 \item Elliott contains other, equivalent definitions of neutrality.
\item I have a necessary and sufficient condition for symplectic neutrality.
\end{itemize}
\subsubsection{Genoni2012 recursivity}
$m=m$, $n=n$, any controls. It is sufficient for Lie group controllability. Note this is not as broad as the neutrality theorem above. Genoni2012.
\subsubsection{Wu non-empty set}
$m=1$, $n=1$, symmetric controls. Necessary and sufficient condition for control on a Lie group. Wu2007 and also p117 of Elliott seems to look very similar.

\subsection{Single input unbounded}
\subsubsection{Jurdjevic and Kupka sufficient condition}
$m=1$, $n=n$, unbounded controls. Sufficient condition for control on a semi-simple? Lie group. Theorem 3.11 in Elliott. Also found in Jurdjevic I think.
\begin{itemize}
 \item This seems like it must be broader than neutrality after fixing the control parameters. If this is true then neutrality is not equivalent to controllability.
\end{itemize}

\subsubsection{El-Assoudi sufficient condition}
$m=1$, $n=n$, unbounded controls. Sufficient condition for control on a semi-simple Lie group. Presumably broader than above; pretty much takes all properties of the symplectic group. ElAssoudi2014. 
\subsection{Huang's theorem}
$m=m$, $n=n$, unbounded controls. Controllability on a homogeneous space of the Lie group in infinite dimensional Hilbert space, intersected with the set of analytic vectors. Huang1983, Wu2005. 
\begin{itemize}
 \item It is the equality in the theorem that is most confusing. Why can it not be higher than the dimension of the manifold?
\end{itemize}

\subsection{Trivial theorems}
\subsubsection{Blow away the drift field}
$m>1$, $n=n$, unbounded controls. Can blow away the drift field and generate the algebra with the control fields, thus making LARC necessary and sufficient. Necessary and sufficient condition for control on a Lie group. JurdjevicGeo.
\subsubsection{Generic Generation}
$m>1$, $n=n$, unbounded controls. Can blow away the drift field. Generators are dense in the algebra and so LARC not required. JurdjevicGeo.

\section{The map of Gaussian states}
This map of bilinear control theory is applied to my map of Gaussian states. Symplectic group manifold of dimension $n(2n+1)$, symplectic vector space, metaplectic group manifold, acting on the Hilbert space with a sub-Kahler manifold: the unit sphere or projective Hilbert space in which is a sub-manifold of pure Gaussian states diffeomorphic to the open convex cone of covariance matrices in dimension $n(n+1)$.

\end{document}